This is the new C++ implementation of the U\+E\+S\+M\+A\+NN modulatory neural network architecture, based on the original code used in my thesis. See \href{http://users.aber.ac.uk/jcf12/research/uesmann/}{\tt my University web site for more details and publications} or the brief introduction below.

\subsection*{What is U\+E\+S\+M\+A\+NN?}

U\+E\+S\+M\+A\+NN is a very simple modification of a standard multilayer perceptron (M\+LP) with a logistic sigmoid activation function, trained using stochastic gradient descent. The modification consists of a modulatory factor {\itshape h} on the weights, such that the weights have their nominal values at {\itshape h}=0 and double those values at {\itshape h}=1. The biases are unmodulated.

As such, the network is able to perform multiple functions at different modulator levels, and is typically trained using examples of one function at {\itshape h}=0 and another at {\itshape h}=1, where it typically performs well. For example, a single network can be trained to perform any possible pairing of binary boolean functions in the same number of network parameters (weights and biases) required for a single such function. It has also been tested in \hyperlink{classMNIST}{M\+N\+I\+ST} handwriting recognition and line recognition tasks, and in a homeostatic robot control problem.

A rather more complete set of documentation, including a description of the network and a Doxygen docs, can be found at

\href{https://jimfinnis.github.io/uesmanncpp/html/index.html}{\tt https\+://jimfinnis.\+github.\+io/uesmanncpp/html/index.\+html}

\subsection*{About the code}

The code itself is very simplistic, using scalar as opposed to matrix operations and no G\+PU acceleration. This is to make it as clear as possible, as befits a reference implementation, and also to match the implementation used in the thesis. There are no dependencies on any libraries beyond those found in a standard C++ install, and libboost-\/test for testing. You may find the code somewhat lacking in modern C++ style because I\textquotesingle{}m an 80\textquotesingle{}s coder.

Implementations of the other network types mentioned in the thesis are also included\+:


\begin{DoxyItemize}
\item output blending (training two networks with identical architectures to perform the two different functions and using the modulator to linearly interpolate between their outputs);
\item h-\/as-\/input (applying the modulator as an extra input to a standard M\+LP and training accordingly)
\item plain (a straightforward M\+LP with no modifications)
\end{DoxyItemize}

I originally intended to use Keras/\+Tensorflow, but would have been limited to using the low-\/level Tensorflow operations because of the somewhat peculiar nature of optimisation in U\+E\+S\+M\+A\+NN\+: we descend the gradient relative to the weights for one function, and the gradient relative to the weights times some constant for the other, alternating between the two. This makes the standard optimisers (such as A\+D\+AM) unsuitable. More investigation is planned, however, because a U\+E\+S\+M\+A\+NN layer may prove useful within a larger system. {\bfseries If you can help with this, please let me know.}

\href{https://travis-ci.com/jimfinnis/uesmanncpp}{\tt } 